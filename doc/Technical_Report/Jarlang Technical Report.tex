\documentclass[twoside,12pt,titlepage,a4paper]{article}
\usepackage{url}
% kentHarvard requires natbib
\usepackage{natbib}
% add line numbers
\usepackage{lineno}
\linenumbers 

\usepackage[pass]{geometry}
\usepackage{graphicx}
\renewcommand{\baselinestretch}{1.3}
\usepackage{todonotes}

\title{CO600 Project: Jarlang\\ Technical Report}
\author{
	\begin{tabular}{ c c c }
		Chris Bailey & Andrew Johnson & Nick Laine \\ 
		\url{cb661@kent.ac.uk} & \url{apj8@kent.ac.uk} & \url{nl227@kent.ac.uk}  
	\end{tabular}\\
	\\ \vspace{10mm}
   \includegraphics[scale=0.6]{Kent_Comp_294_RGB} \\
   School of Computing \\
   University of Kent \\
   United Kingdom \\ \vspace{10mm} \\ Word Count: 6,100}
\begin{document}

\newgeometry{hmarginratio=1:1}    %% make layout symmetric
\maketitle
\restoregeometry              %% restore the layout

\begin{abstract}
This paper describes the development and operation of an Erlang to JavaScript command-line transpiler, referred to as Jarlang. The aim of this transpiler is to generate human-readable valid JavaScript from any error-free Erlang module source code, in a format that can be included in a web page and integrated into the front end code environment with ease. In this objective the project has succeeded, though the function of the output JavaScript is difficult to test without also transpiling the required standard library modules, which we have not had time to complete.
\todo{Abstract is a bit short: Aim for 100-200 words, currently aprox:90.}
\end{abstract}

\section{Introduction}
\label{Introduction}
	\todo{Introduction is a slightly modified copy of the abstract assessment text, will need editing \& maybe total re-write.}
	The central aim of this project has been to write an Erlang to ES6 JavaScript transpiler, referred to as Jarlang, enabling solutions to problems in JavaScript to be expressed as Erlang code. Jarlang also enables websites to unify the languages used in their stack, utilising Jarlang-transpiled Erlang on the frontend, and either Erlang or Jarlang-transpiled Erlang (running via Node.js) on the backend.
	
	Our transpiler is written in Erlang, the same language as our source language. The motivation behind this is so that in time we can bootstrap the Jarlang transpiler and transpile it to JavaScript, thus enabling us to compile or interpret raw Erlang in a browser environment independent of any Erlang packages installed on a host’s system. Due to time constraints we have had to rely on the Erlang compiler to validate, optimize and parse the Erlang source code, and as the compiler is not written in Erlang this aspect of the project was subsequently deemed too ambitious. We also intended to, and succeeded in, implementing as much of Erlang as possible in JavaScript, including function overloading, modules, tail recursion and ultimately – and more ambitiously – emulating Erlang-style concurrency using JavaScript’s event loop.
	
	The bulk of our code is written in Erlang and utilises the Erlang compiler to generate Core Erlang, an intermediate language used in Erlang’s compilation process. We then use built-in functions to get the Core Erlang abstract syntax tree which is then translated into a valid JavaScript abstract syntax tree via a series of transformation steps. Valid JavaScript is then derived from this abstract syntax tree via the employment of existing JavaScript projects such as escodegen.
	
\section{Background}
\label{Background}


\section{Aims}
\label{Aims}


\section{Project Results}
\label{Results}

\todo{Excerpt from guideline doc:	(Several technical content sections)
	This is where you go into detail about what you have done. You will need to decide the titles for these sections yourself; they will depend on the content of the project. These sections should summarise the technical and scientific achievements of the project.
	
	Depending on the nature of your project, these sections may include: a comparison of different approaches that you considered, accounts of experimental work, mathematical analyses, specifications, top-level architectural diagrams, results obtained, problems encountered, workarounds, user evaluations, performance measures, testing regimes and results, comparisons between different approaches adopted, comparisons with existing work on similar problems.
	
	In particular, you should give a mixture of general discussion of your work and particular examples. Too much general discussion and the reader cannot easily get a handle on what you are doing; too many specific examples and the document fails to "tell a story".
}

\todo{This text is a copy of the abstract assessment text, it will need re-writing.}
The Jarlang transpiler is capable of successfully transpiling relatively simple Erlang source code to JavaScript. One of the main issues Jarlang has is the lack of a standard library, thus more complex code may not run after being transpiled. In time, this issue will be resolved by simply implementing the necessary Erlang language features (of particular importance, Erlang’s data types) in JavaScript, so that we can transpile more of Erlang’s standard library to utilise in transpiled code. We have succeeded in implementing the majority of Erlang’s data types, however they have yet to be integrated into the transpiled code.

In addition to this, we have implemented actor-style concurrency in JavaScript. Our success in implementing this has eliminated many of the initial concerns we harboured for this project, however this has raised other concerns such as garbage collection of completed processes. We have decided that this is outside the scope of this project.

\subsection{Erlang Datatypes}
\subsection{AST Conversion}
\subsubsection{Compiler Optimisations}
\subsubsection{Pattern Matching, Conditional Logic \& Message Receiving}
\subsubsection{Variable Declaration \& Assignment in Matching}
\subsection{Jarlang Run-Time System}
\subsection{ErlPKG}

\section{Conclusions}
\label{Conclusions}

 [section not done yet]

\section{Acknowledgement}
The authors wish to thank Scott Owens for his assistance
and advice.

\appendix
\section*{Appendix A. TODO: Do we need an appendix?}


[section ommitted]



\vskip 0.2in
\todo{Bibliography is still the template version, will need replacing.}
\bibliography{jarlang_technical_report}
\bibliographystyle{kentHarvard}

\end{document}
