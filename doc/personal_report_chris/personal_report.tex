\documentclass[twoside,12pt,titlepage,a4paper]{article}
\usepackage{url}
% kentHarvard requires natbib
\usepackage{natbib}
% add line numbers
\usepackage{lineno}
%\linenumbers %Comment this out to remove line numbers

\usepackage[pass]{geometry}
\usepackage{graphicx}
\renewcommand{\baselinestretch}{1.3}
\usepackage{todonotes}
%\usepackage{minted}
\usepackage{subcaption}
\usepackage{afterpage}
\usepackage{float}

\newcommand\blankpage{%
    \null
    \thispagestyle{empty}%
    \addtocounter{page}{-1}%
    \newpage}

% Fix weird page margins?
\setlength{\oddsidemargin}{1cm}
\setlength{\evensidemargin}{1cm}
\setcounter{secnumdepth}{3}
\raggedbottom
\frenchspacing

\title{CO600 Project: Jarlang\\ Personal Report}
\author{
	\begin{tabular}{ c c c }
		Chris Bailey\\
		\url{cb661@kent.ac.uk}
	\end{tabular}\\
	\\ \vspace{10mm}
		\includegraphics[scale=0.6]{Kent_Comp_294_RGB} \\
		School of Computing \\
		University of Kent \\
		United Kingdom \\ \vspace{10mm} \\ Word Count: 1698 (via texcount)}
\begin{document}

\newgeometry{hmarginratio=1:1}    %% make layout symmetric
\maketitle
\restoregeometry              %% restore the layout

\section{Project Contribution}
\label{Project Contribution}
	Throughout the project, it was primarily Andrew working on the Erlang codebase; implementing and refining
	our abstract-syntax tree translation module which quite literally is the main functionality of Jarlang as a compiler. On
	the other hand, Nick's contribution to the project was spent almost entirely in implementing the Jarlang runtime environment 
	(our JavaScript codebase) so that we could actually execute the result of Andrew's work.

	My contributions to the Jarlang project as a whole however, have been more or less equally distributed between both codebases,
	in both actually implementing features or bugfixes, as well as ideating and helping both Nick and Andrew acheive what they have.
	I will outline my main contributions below:

	\subsection{Project Architecture}
		One of the most notable contributions I've made to this project as a whole involve the several architectural decisions
		I've made in order to make this project as clean, extensible, testible and simple as possible.

		I was the one who originally instigated the project, as well as the one who first started writing code for the project 
		and as such paved the way for the initial direction we took. In addition to this,
		the majority of the research done for this project (i.e. finding other similar projects such as LuvvieScript, as
		well as finding out about useful tools which aided in our AST generation / design) was done by me which laid a
		strong foundation for future development.

		As the one in our group who multitasked most, working on both the Erlang and JavaScript codebases, whenever I noticed
		something get very messy, or unneccessarily complex, I would call for a refactoring to make our codebases nicer,
		more extensible and better designed. Whilst these refactorings took valuable time away
		from functionality implementation, they made future functionality implementation much easier to perform. 

		This happened many times across both our codebases. For example, I called for a heavy refactoring of Andrew's AST translation
		module when he needed help and his code was indecipherable in terms of the complex nested pattern matching he used to determine
		what functions needed to be called in order to perform translation. We ended up refactoring that to switch into several much
		smaller functions based primarily on the nodetype of a given AST node ultimately allowing us to easily add more features in terms
		of things we support translation of, as well as allowing me and Nick to help out if need be.

		Another example in our JavaScript codebase is the fact that we were originally creating classes for our JavaScript
		representations of Erlang datatypes with the ES5 closure method, which works very nicely but gives us some drawbacks.
		The main drawback this faced was discovered when we wanted all of our datatypes to share an interface to allow the runtime system
		to assume things about all types (i.e. how to read the value of a type, how to stringify it etc). 
		The most ideal and most elegant solution to this was to refactor our codebase to use ES6 Class constructors which allow us to use the 
		'extends' keyword --- allowing all datatypes to inherit from a base class which not only perscribes a standard interface but also
		allows us to easily implement datatype matching via simple overridding of inherited methods and values.

		Thus at a high level, one could say that I was responsible for our project following good software engineering practices,
		as well as consulting, commenting on and improving on our codebase iteratively throughout the project duration. This not
		only extends to the JavaScript or Erlang codebase but includes me taking it upon myself to set up a decent toolchain to 
		build Jarlang, including writing Erlpkg (discussed below) and writing the initial makefiles and shell scripts needed
		to build Jarlang easily from scratch.

	\subsection{The Jarlang Compiler}
		As well as architectural / software engineering based contributions, I also worked on several features in the Jarlang
		compiler which follow.

		Being the one who started our Erlang codebase, I was the one who researched methods of hooking into the Erlang compiler
		and extracting from it our CoreErlang AST which our project depends on. In addition to this, I also wrote an open source
		tool, Erlpkg, which not only builds binaries for Erlang projects, but also generates help text for good UX as well as
		parses command line options for users allowing us to support plenty of quality-of-life features such as outputting
		our intermediate compilation steps or toggling multithreaded compilation for better Erlang errors.

		I was also the one who implemented module compilation in the Jarlang compiler, so whilst Andrew did the much more difficult
		task of compiling functions into JavaScript, I implemented code which took functions and a module prototype and created
		a correct JavaScript module out of these. To aide with this I implemented an interface for generating JSON nodes which represent 
		JavaScript AST nodes which is used extensively by the core of the compiler.

		Lastly, we all originally had wildly different programming styles and as such, over time, I enforced a common style guide
		for our Erlang modules such that it was easy and regular to read no matter what module you were reading. I integrated
		Dialyzer (a static analyzer for Erlang), as well as Elvis (a linter for Erlang) to aide with this.

	\subsection{The Jarlang Runtime}
		During the course of the project, I not only architected how we should implement the runtime system, but I also actively took
		a role in implementing parts of it.

		The segments of the Jarlang runtime I implemented primarily centered around the actor-model features we offer including
		writing the actor agents themselves, implementing message passing functions to allow actors to communicate, as well as
		implementing the PID and Process datatypes such that PIDs would be unique identifiers for Processes, which would run asynchronously
		and encapsulate Erlang style concurrent behaviour.

		In addition to this, whilst Nick primarily worked on designing the other datatypes Jarlang uses, as well as implementing much of
		the base modules currently integrated into the runtime (IO and Erlang modules), I implemented the core of the runtime including
		emulation of the process queue, atom table, and ETS table. I also implemented just about anything which needs to reason or
		work with processes as outside of a few key places, our concurrency abstraction is completely transparent.

		I also implemented the layers responsible for interopability between the JavaScript runtime and our Jarlang runtime.


\section{Project Reflection and Evaluation}
\label{Project Reflection and Evaluation}
	In retrospect, I personally think we as a group did many things right for Jarlang. I feel as though Jarlang in its current state is
	at least functional and can successfully compile and execute Erlang code on/for the web.

	There are a few things I'd change if given the chance to go back and start the project from scratch again though. As the person
	solely responsible for most of the architectural decisions we've made, I admit that it took us far too long to get to a point
	where we worked under best practices for Erlang programming --- not only did we try to write our own typechecker, we completely
	failed to successfully test our code or utilise Erlang tools such as dialyzer until halfway through the project.

	In addition to this, we ended up refactoring our code a lot, which whilst not a bad thing perse, definitely did waste time I would
	have liked to dedicate to implementing new features. This is because whilst everyone on our team is a strong programmer with a diverse
	range of expertise, everyone was implementing things by themselves before pushing their changes to our Git repository, which means that
	it was only after features were fully implemented that we were able to critically comment on them.

	If we had utilised more pair-programming or at least utilised branches more, we would be able to either critically comment on
	how things were being implemented before it would require a large refactor, or be able to push more incremental updates to our
	Git repo without worrying about breaking anything, allowing other team members to look at our changes in more incremental steps.

	Because we failed to utilise Git branches successfully, and lack any sort of continuous integration or testing
	framework, often times we would end up breaking features which previously worked, only setting us back for swathes of time if
	the feature being broken wasn't noticed posthaste. 
	
	Despite having makefile targets for dialyzer, linting and testing, 
	very few tests were written and most of the time my colleagues wouldn't remember to lint or run dialyzer before asking for help
	or trying to debug blind. Our JavaScript code also didn't have any tests or even a bundler until very recently, which did
	slow down development.

	Though critically speaking, I personally do believe our project has reached at very least what I wanted out of it at the beginning
	of the year; when I proposed the project to Nick and Andrew, I thought it was an insane idea which was completely out of reach. 
	We all approached the year ready to take the outcome of the project, no matter what, as a huge learning experience and we simply
	gave it everything we thought it needed which seemingly paid off.

	Despite parts of it not functioning exactly as we want it yet, as well as the likely possibility of there being bugs in our codebase
	from simple lack of testing and lack of knowledge, I would still say that from an academic perspective as well as a pragmatic 
	one, our year spent developing Jarlang has been successful.

\section{The Future}
\label{The Future}
	There is a common consensus amongst our group that whilst we recognise the current flaws of our project, we are all rather proud
	of where the project has gotten in such short a time. We are however, obviously dissapointed that we haven't implemented everything
	to a quality which we'd like --- as stated, there are quite a few things I'd personally change if given a chance to restart --- but
	this does not mean that our project will statically remain how it is.

	At very least, me and Andrew have agreed that after the conclusion of this project, we both want to continue working on maintaining Jarlang
	as an opensource project free for anyone to use. This includes not only bugfixes but also continued implementation of Jarlang for
	the forseeable future, which I can say is honestly quite exciting as I genuinely enjoy working on this project.

%\vskip 0.2in
\bibliographystyle{kentHarvard}
%\bibliographystyle{plain}
\bibliography{personal_reportz}
\blankpage
\end{document}
